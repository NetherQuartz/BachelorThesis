%% Преамбула TeX-файла

% 1. Стиль и язык
\documentclass[utf8x, 14pt]{G7-32} % Стиль (по умолчанию будет 14pt)

% Остальные стандартные настройки убраны в preamble.inc.tex.
\include{preamble.inc}
\include{listings.inc}

\usepackage{fontspec}
\setmainfont{Times New Roman}

% Настройки листингов.
\usepackage{local-minted}

% Полезные макросы листингов.
\include{macros.inc}


\begin{document}

\frontmatter % выключает нумерацию ВСЕГО; здесь начинаются ненумерованные главы: реферат, введение, глоссарий, сокращения и прочее.

\includepdf[pages=-]{../inc/title.pdf}

\Referat

Отчёт содержит \pageref{LastPage}\,страницу%
\ifnum \totfig >0
, \totfig~рисунков%
\fi
% \ifnum \tottab >0
% , \tottab~таблицы%
% \fi
, 1~таблицу%
%
\ifnum \totbib >0
, \totbib~источников%
\fi
%
\ifnum \totapp >0
, \totapp~приложение.%
\else
.%
\fi

ГЕНЕРАЦИЯ ТЕКСТА, НЕЙРОННЫЕ СЕТИ, NLP, TRANSFORMER, ДООБУЧЕНИЕ, RUGPT-3

В работе представлено решение задачи дообучения нейросетевой языковой модели ruGPT-3 Small архитектуры Transformer на сравнительно небольшом корпусе текстов, принадлежащих конкретной предметной области, для усвоения моделью стилистики текстов данной области и последующего её встраивания в графический веб-интерфейс пользователя, предоставляющий возможности по генерации новых текстов, принадлежащих той же предметной области.


\tableofcontents

\printnomenclature % Автоматический список сокращений

\Introduction
\label{cha:intro}

В современном мире с ростом уровня образования и увеличением спроса на специалистов высокой квалификации для оптимизации производственных процессов требуется повышать уровень автоматизации предприятий, чтобы разгрузить работников и предоставить им возможность заниматься интеллектуальным трудом. А с увеличением вычислительных мощностей и развитием компьютерных наук всё больше рутинных задач поддаются автоматизации. Задача автоматизации написания разного рода текстов стоит особенно остро, так как она актуальна в самых разных сферах человеческой деятельности.

В настоящей работе предлагается система, которая позволяет автоматизировать большую часть действий, требуемых для создания базовой структуры документа и его частичного заполнения, предоставляя возможность коррекции и дополнения полученного текста пользователем <<на лету>>.

Работа системы демонстрируется применительно к задаче генерации сценариев юмористических телешоу. Актуальность решения этой задачи обусловлена его применимостью в сферах психологии и психиатрии для тестирования уровня эмпатии способом, близким к методике А. Меграбяна и Н. Эпштейна \cite{art:psy_test}: пациенту предлагается прочитать несколько текстов и ответить на ряд вопросов, касающихся испытываемых им чувств, после чего на основании полученных ответов делается вывод по поводу уровня его эмпатии.

Но возможности данной системы не ограничены только этой предметной областью. Предлагаемое решение предоставляет гибкий механизм дообучения под генерацию текстов из той предметной области, которой принадлежит обучающая выборка текстов.
\vspace*{\fill}
\addcontentsline{toc}{chapter}{Основная часть}
\begin{center}
    \bfseries\MakeUppercase{Основная часть}
\end{center}
\label{cha:main}
\vspace*{\fill}

\mainmatter % это включает нумерацию глав и секций в документе ниже

\chapter{Теоретическая часть}
\label{cha:theory}

\section{Определения}

\textbf{Корпус текстов} --- множество подобранных и определённым образом обработанных текстов.

\textbf{Токен} --- элементарная единица разбиения корпуса.

\textbf{Токенизация} --- процесс разбиения корпуса на токены с присвоением им уникальных числовых идентификаторов.

\textbf{Языковая модель} --- распределение $P(w_t | w_1,w_2,w_3,\dots,w_n)$ вероятностей встретить токен $w_t$ в корпусе сразу после $n$ токенов $w_i$, $i\in[1, n]$, идущих подряд, где $w_i \in W \, \forall i$, $W$ --- множество всех токенов корпуса, $n$ --- длина контекста модели.

\textbf{Длина контекста} --- количество $n$ токенов $w_i$, $i\in[0, n]$, предшествующих токену $w_t$, от которых зависит вероятность появления в тексте токена $w_t$.

\textbf{Дообучение} --- процесс обучения уже обученной на некоторых данных модели машинного обучения на новых данных. В случае языковой модели это означает подстройку модели под новое распределение токенов.

\textbf{Перплексия} --- мера схожести двух вероятностных распределений, используемая для оценки качества генерации текста языковой моделью. Перплексия задаётся формулой \ref*{eq:perplexity}.
\begin{equation}
    \label{eq:perplexity}
    \textrm{PP}(W)=\sqrt[n]{\frac{1}{P(w_1,w_2,\dots,w_n)}}
\end{equation}

\section{Постановка задачи}

Заданы:
\begin{itemize}
    \item корпус, состоящий из текстов, принадлежащих конкретной предметной области,
    \item предобученная нейросетевая языковая модель, хорошо моделирующая вероятностное распределение слов в естественном языке.
\end{itemize}
Требуется: дообучить данную языковую модель на данных из корпуса, получив новую языковую модель, моделирующую распределение вероятностей слов в данном корпусе, и применить её для генерации новых текстов, принадлежащих предметной области данного корпуса, встроив в графический пользовательский интерфейс.


\backmatter %% Здесь заканчивается нумерованная часть документа и начинаются ссылки и
            
\Conclusion
\label{cha:conclusion}

Приходим к выводу…

\blindtext%% заключение


\bibliographystyle{ugost2008}
\bibliography{thesis}



\appendix   % Тут идут приложения

\chapter{Листинги исходного кода}
\label{cha:appendix1-listing}

\lstinputlisting[language=Python, caption={Графический интерфейс пользователя}, label=lst:demo]{../inc/code/demo.py}
\lstinputlisting[language=Python, caption={Модуль генерации текста}, label=lst:generate]{../inc/code/generate.py}
\lstinputlisting[language=Python, caption={Скрипт для валидации и обработки данных}, label=lst:humanize]{../inc/code/humanize_data.py}


\end{document}

%%% Local Variables:
%%% mode: latex
%%% TeX-master: t
%%% End:
