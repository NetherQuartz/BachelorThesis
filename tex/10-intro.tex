\Introduction
\label{cha:intro}

В современном мире с ростом уровня образования и увеличением спроса на специалистов высокой квалификации для оптимизации производственных процессов требуется повышать уровень автоматизации предприятий, чтобы разгрузить работников и предоставить им возможность заниматься интеллектуальным трудом. А с увеличением вычислительных мощностей и развитием компьютерных наук всё больше рутинных задач поддаются автоматизации. Задача автоматизации написания разного рода текстов стоит особенно остро, так как она актуальна в самых разных сферах человеческой деятельности.

В настоящей работе предлагается система, которая позволяет автоматизировать большую часть действий, требуемых для создания базовой структуры документа и его частичного заполнения, предоставляя возможность коррекции и дополнения полученного текста пользователем <<на лету>>.

Работа системы демонстрируется применительно к задаче генерации сценариев юмористических телешоу. Актуальность решения этой задачи обусловлена его применимостью в сферах психологии и психиатрии для тестирования уровня эмпатии способом, близким к методике А. Меграбяна и Н. Эпштейна \cite{art:psy_test}: пациенту предлагается прочитать несколько текстов и ответить на ряд вопросов, касающихся испытываемых им чувств, после чего на основании полученных ответов делается вывод по поводу уровня его эмпатии.

Но возможности данной системы не ограничены только этой предметной областью. Предлагаемое решение предоставляет гибкий механизм дообучения под генерацию текстов из той предметной области, которой принадлежит обучающая выборка текстов.