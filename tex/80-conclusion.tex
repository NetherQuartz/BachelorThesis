\Conclusion
\label{cha:conclusion}

В результате выполнения описанной работы были решены следующие подзадачи:
\begin{itemize}
    \item найден и предобработан корпус --- набор сценариев юмористических телешоу,
    \item выбрана и дообучена предобученная нейросетевая языковая модель ruGPT-3 Small архитектуры Transformer с использованием библиотек для глубокого обучения PyTorch и Transformers для языка Python,
    \item разработан графический веб-интерфейс с использованием фреймворка Streamlit на Python, предоставляющий следующие возможности:
    \begin{itemize}
        \item дополнение введённого пользователем текста сгенерированным нейронной сетью,
        \item отмена результатов генерации,
        \item свободное редактирование получившегося документа,
        \item сохранение результата на компьютере,
    \end{itemize}
    \item приложение для упрощения дистрибуции упаковано в изолированное окружение --- контейнер, созданный при помощи технологии Docker.
\end{itemize}

По итогу проделанной работы можно заключить, что все изначально поставленные задачи были успешно выполнены:
\begin{itemize}
    \item сгенерированные моделью тексты по форме действительно являются сценариями юмористических телешоу: они имеют аналогичную структуру, а при прочтении человеком могут вызвать у него смех,
    \item графический интерфейс получился удобным для взаимодействия с моделью, предоставляет достаточно гибкие возможности для экспериментов,
    \item технология Docker как средство упаковки приложения оказала положительное влияние на простоту развёртывания приложения на локальном компьютере, позволив прописать чёткие и универсальные инструкции для пользователя.
\end{itemize}
